\subsubsection{Simple Text Visual Changing} % Nazwa pod-podsekcji

% https://tex.stackexchange.com/questions/138/what-are-underfull-hboxes-and-vboxes-and-how-can-i-get-rid-of-them
regular, regular, \emph{distinguished}, regular, \textbf{bold text}, regular, \texttt{machine font}, \textit{ italic, {\em distinguished text among italics}, italic}, \textsc{capital}, \textsf{non-sheriff}

% Różnie rodzenie fontów pokazane powyżej. Dodatkowo sam rozmiar / rodzaj fontów można ustawiać w pliku konfiguracyjnym .cls

\subsubsection{Simple Text Size Changing}

% Zabawa rozmiarami fontów
{\small small font}, {\large large font}, average size, {\footnotesize footnote size}, {\Large very large font size},
\textit{\LARGE even larger font (with italic)} {\huge and huuuuuge} {\Huge and HUUUUUGER}

\subsubsection{Commands}

%\begin{nazwa-polecenia}
%tekst
%\end{nazwa-polecenia}

Points:: % Po kolei prezentacja punktowania rzeczy. Wszystko zaczyna się od \begin i kończy na \end. Każdy nowy element zaczyna się nowym punktem (\item). Wewnątrz punktów można używać innych rzeczy - tu zaprezentowane flushowanie tekstu
\begin{itemize}
    \item \textbf{Flushing:}
        \begin{flushleft}
        Left flush
        \end{flushleft}
        \begin{center}
        Center
        \end{center}
        \begin{flushright}
        Right flush.
        \end{flushright}
    \item Some other item -- with some desription
    \item And you know what?
        \begin{itemize} % Wyliczanki można zagnieżdżać
        \item INCEPTION
        \end{itemize}
\end{itemize}

And Enumerations: % No i poza itemize jest jeszcze enumerate czyli wyliczenia
\begin{enumerate}
    \item First point
    \item Second point
    \item Third point
    \item And inception again:
    \begin{enumerate}
        \item Item new
    \end{enumerate}
\end{enumerate}