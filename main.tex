\documentclass[12pt]{tech} 

\usepackage[utf8]{inputenc}
\usepackage{polski}
\usepackage[linesnumbered,ruled,vlined]{algorithm2e}
\usepackage{mathtools}
\usepackage{amsmath}
\usepackage{array}
\usepackage{pbox}
\usepackage{sudoku}
\usepackage{url}
\def\UrlBreaks{\do\/\do-}
\usepackage{breakurl}
\usepackage[breaklinks]{hyperref}
\setcounter{secnumdepth}{3}

\begin{document} % Zaczynamy dokument

\begin{titlepage}

    \begin{center}
    
    \includegraphics[scale=0.08]{Images/logo_wsei.jpg}
    %\includegraphics[scale=1.25]{agh_nzw_s_en_1w_wbr_pms}\\[0.2cm]
    
    \textsc{Wyższa Szkoła Ekonomii i Informatyki}\\[0.2cm]
    \vspace{2cm}
    \textsc{Katedra Informatyki}\\[1cm]
    
    {\huge \bfseries Outbreak}\\[1cm]
    
    \textbf{Dokumentacja techniczna}\\[1cm]
    
    \vfill
    
    \begin{minipage}{0.8\textwidth}
    \begin{flushleft}
    {\large \emph{Imię i nazwisko:} } { \hfill Imię  \textsc{Nazwisko}} \newline
    \end{flushleft}
    \end{minipage}\\[2cm]
    
    \large{\today}
    
    \end{center}
    
    \end{titlepage}
     % wrzucamy zawartość strony title

\thispagestyle{empty}

% spis tresci
% spis obrazkow
% spis code-snippetow

\newpage % dodajemy nową stronę
\tableofcontents % generujemy spis sekcji, subsekcji i subsubsekcji
\newpage % nowa strona
\listoffigures % generujemy listę obrazków
\newpage
\listoftables % generujemy listę tabelek
\newpage

\section{Wstęp} % Nazwa sekcji

\section{TEST}
% \input{ne}
\subsection{Some Simple stuff} % Nazwa podsekcji
\subsubsection{Simple Text Visual Changing} % Nazwa pod-podsekcji

% https://tex.stackexchange.com/questions/138/what-are-underfull-hboxes-and-vboxes-and-how-can-i-get-rid-of-them
regular, regular, \emph{distinguished}, regular, \textbf{bold text}, regular, \texttt{machine font}, \textit{ italic, {\em distinguished text among italics}, italic}, \textsc{capital}, \textsf{non-sheriff}

% Różnie rodzenie fontów pokazane powyżej. Dodatkowo sam rozmiar / rodzaj fontów można ustawiać w pliku konfiguracyjnym .cls

\subsubsection{Simple Text Size Changing}

% Zabawa rozmiarami fontów
{\small small font}, {\large large font}, average size, {\footnotesize footnote size}, {\Large very large font size},
\textit{\LARGE even larger font (with italic)} {\huge and huuuuuge} {\Huge and HUUUUUGER}

\subsubsection{Commands}

%\begin{nazwa-polecenia}
%tekst
%\end{nazwa-polecenia}

Points:: % Po kolei prezentacja punktowania rzeczy. Wszystko zaczyna się od \begin i kończy na \end. Każdy nowy element zaczyna się nowym punktem (\item). Wewnątrz punktów można używać innych rzeczy - tu zaprezentowane flushowanie tekstu
\begin{itemize}
    \item \textbf{Flushing:}
        \begin{flushleft}
        Left flush
        \end{flushleft}
        \begin{center}
        Center
        \end{center}
        \begin{flushright}
        Right flush.
        \end{flushright}
    \item Some other item -- with some desription
    \item And you know what?
        \begin{itemize} % Wyliczanki można zagnieżdżać
        \item INCEPTION
        \end{itemize}
\end{itemize}

And Enumerations: % No i poza itemize jest jeszcze enumerate czyli wyliczenia
\begin{enumerate}
    \item First point
    \item Second point
    \item Third point
    \item And inception again:
    \begin{enumerate}
        \item Item new
    \end{enumerate}
\end{enumerate} % Wrzucenie w to miejsce zawartości pliku Input1.tex

\subsection{Labels}
\subsubsection{Label}
\label{label:label_title} % Każda sekcja, podsekcja, tabela, obrazek etc może mieć swój własny \label -> Unique ID

% Ten unique ID może mieć odsyłacz w tekście. LaTeX sam ogarnie numerowanie oraz hiperłącza do tego - pdf będzie mega łatwy do czytania i poruszania się po nim
In chapter \ref{label:label_title} we can reference a label. Apart from that we can also add and image as shown on the figure \ref{fig:example_figure}.
    
\begin{figure}[h] % begin dla jakiegoś obrazka - parametry w nawiasach są różne: https://www.overleaf.com/learn/latex/Inserting_Images
    \centering
    \includegraphics[scale=0.75]{Images/WSEI-Logo.png} % pod powyższym linkiem są również wyjaśnione dodatkowe parametry do obrazków ( to co wewnątrz [] )
    \caption[Short text]{This is a figure caption.} % Opis obrazka - w pierwszych wąsach jest wpisywany do listoffigures, w drugich opis, który się wyświetla pod samym obrazkiem
    \label{fig:example_figure} % no i znowu labelka
\end{figure}

Other stuff will be presented in section \ref{label:table_title}

\subsubsection{Table}
\label{label:table_title}

In this chapter we'll create a table:
\begin{table}[h] %te parametry wrzucania ([h] są takie same jak dla obrazków - link wyżej
    \centering
    \begin{tabular}{|c|l||c|} % https://www.overleaf.com/learn/latex/Tables - a tu są wyjaśnione te dziwaczne rzeczy w {} nawiasach - parametry co do wielkości i formatowania tekstu w konkretnej kolumnie
        \hline % linia oddzielająca wiersze
        Some first column data & Second column & Third column \\ \hline \hline
        And new data & \textbf{BOLD!} & New input \\ % i & oddzielenia konkretnych kolumn
        \hline
    \end{tabular}
    \caption[First table.]{Table with filler data} % nazwa tabelki - tak jak dla obrazków powyżej
    \label{tab:first_table} % no i labelka - unikatowa
\end{table}

\begin{table}[h] % inny przykład tabelki
\centering
\begin{tabular}{ | m{5em} | m{1cm}| m{1cm} | } 
  \hline
  cell1 dummy text dummy text dummy text& cell2 & cell3 \\ 
  \hline
  cell1 dummy text dummy text dummy text & cell5 & cell6 \\ 
  \hline
  cell7 & cell8 & cell9 \\ 
  \hline
\end{tabular}
\caption[Second table.]{Second Table with filler data}
\label{tab:second_table}
\end{table}

In table \ref{tab:first_table} we show some simple data. Table \ref{tab:second_table} shows some other approach. And table \ref{tab:third_table} shows other possibilities. % i kilka razy \ref, żeby pokazać, że odwoływanie się do tabelek śmiga

\begin{table}[!htb]
\centering
\begin{tabular}{|c|c|c|c|c|c|c|c|}
\cline{2-8}
\multicolumn{1}{l|}{} & \multicolumn{7}{|c|}{Numbers} \\ \hline
Numbers & 3/4 & 5 & 6 & 7/8 & 15/16 & 17 & 18 \\ \hline
Numbers & + & + & & & & & \\ \hline
Numbers & -- & & + & o+ & & & \\ \hline
\end{tabular}
\caption{Third table.}
\label{tab:third_table}
\end{table}

In table \ref{tab:third_table} multicolumn is used to join some columns together. Cline creates some lines like hline but only in selected columns.

\subsubsection{Equations}
\label{label:equations_title}

Here you can see how you should write equation.
\begin{equation} % dodatkowo w latexie dziecinnie prosto można sobie robić wzorki - tak jak wszystko jest \begin oraz \end. Jak korzystamy z begin / end equation to od razu mamy automatyczną numerację wszystkich wzorów jak i możliwość dodawania labelek
    G = mc^2
    \label{eq:equation_one}
\end{equation}
And second one:
\begin{equation}
    x = \frac{\sqrt{a^2-2ab+b^2}}{a-b}
    \label{eq:equation_two}
\end{equation}
Where obviously $a - b \ne 0$. % wzory można również dodawać zamykając je wewnątrz $ $ - tu nie ma numeracji tak ogarniętej jak przy equation jak i nie ma labelkowania łatwego

And there are even more equations:
$$A \cup B = \{\, x \colon (x \in A) \vee (x \in B)\,\}$$
$$\sum_{k=1}^{\infty}\frac{1}{k^2+1}$$
$$\int_{c}^{d} \left[ \int_{u(y)}^{v(y)} f(x,y)dx \right] dy$$
$$\lim_{n \to \infty} \sum_{k = 1}^n \frac{1}{n} = 0$$
$$\sqrt{2}\sqrt{x^3 + \sqrt{1 + \sqrt{\sqrt{3}-1}}}$$
$$\frac{1}{1 + \frac{1}{1 + \frac{1}{1 + \frac{1}{1 + \dots}}}}$$

\subsubsection{Code snippets}
\label{label:code_title}

And lastly we have some possibilities to add code snippets! % I można dodwać kodzik -> sam wygląd jest skonfigurowany w pliku .cls
\begin{lstlisting}[language=Python]
import numpy as np
    
def incmatrix(genl1,genl2):
    m = len(genl1)
    n = len(genl2)
    M = None #to become the incidence matrix
    VT = np.zeros((n*m,1), int)  #dummy variable
    
    #compute the bitwise xor matrix
    M1 = bitxormatrix(genl1)
    M2 = np.triu(bitxormatrix(genl2),1) 

    for i in range(m-1):
        for j in range(i+1, m):
            [r,c] = np.where(M2 == M1[i,j])
            for k in range(len(r)):
                VT[(i)*n + r[k]] = 1;
                VT[(i)*n + c[k]] = 1;
                VT[(j)*n + r[k]] = 1;
                VT[(j)*n + c[k]] = 1;
                
                if M is None:
                    M = np.copy(VT)
                else:
                    M = np.concatenate((M, VT), 1)
                
                VT = np.zeros((n*m,1), int)
    
    return M
\end{lstlisting}

\begin{lstlisting}[language={[Sharp]C}, caption={C\# exaple}, label={listing:script}]
class Program
{
    public static int CalculateSumRecursively(int n, int m)
    {
        int sum = n;
        if(n < m)
        {
            n++;
            return sum += CalculateSumRecursively(n, m);
        }
        return sum;
   }
    static void Main(string[] args)
    {
        Console.WriteLine("Enter number n: ");
        int n = Convert.ToInt32(Console.ReadLine());
        Console.WriteLine("Enter number m: ");
        int m = Convert.ToInt32(Console.ReadLine());
        int sum = CalculateSumRecursively(n, m);
        Console.WriteLine(sum);
        Console.ReadKey();
    }
}
\end{lstlisting}

As {\color{red}you} can see on listing \ref{listing:script} we can format it and even reference in scripts!


\section{Last Chapter}
\subsection{Cite}
We can also cite some things from bibliography. I want to cite: \cite{sim_models}\cite{accidents}\cite{aggr_timid_driv}





\newpage
\nocite{korki, twoLane, sotl2, sim_models, nasch, sotl, knospe, ris, reducing, bdi, vms, latent, greedy_traffic, makro, rl_lights, busnaschr, itsumo, transims, vdr, intelligent_tlc, bl, korki_2, aggr_timid_driv, measuring, accidents, wekabook, ql_watkins} % tutaj tylko spis wykorzystywanych ID z bibliografii (więcej tu: https://bibtex.eu) - do tego polecam również google scholar i automatyczne branie BibteXa po kliknięciu Cite wyszukiwanego artykułu

%Placeholders:
\bibliography{bibliografia} % wygenerowana bibliografia na podstawie pliku .bib

\end{document}
